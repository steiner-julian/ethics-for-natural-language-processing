\documentclass[a4 paper]{article}
% Set target color model to RGB
\usepackage[inner=2.0cm,outer=2.0cm,top=2.5cm,bottom=2.5cm]{geometry}
\usepackage{setspace}
\usepackage[rgb]{xcolor}
\usepackage{verbatim}
\usepackage{subcaption}
\usepackage{amsgen,amsmath,amstext,amsbsy,amsopn,tikz,amssymb}
\usepackage{fancyhdr}
\usepackage[colorlinks=true, urlcolor=blue,  linkcolor=blue, citecolor=blue]{hyperref}
\usepackage[colorinlistoftodos]{todonotes}
\usepackage{rotating}
\usepackage{enumitem}
%\usetikzlibrary{through,backgrounds}
\hypersetup{%
pdfauthor={Aishik Mandal},%
pdftitle={Homework 0},%
pdfkeywords={Tikz,latex,bootstrap,uncertaintes},%
pdfcreator={PDFLaTeX},%
pdfproducer={PDFLaTeX},%
}
%\usetikzlibrary{shadows}
% \usepackage[francais]{babel}
\usepackage{booktabs}
% \input{macros.tex}

\usepackage{float}

\newlength{\Lnote}
\newcommand{\notte}[1]
     {\addtolength{\leftmargini}{4em}
        \settowidth{\Lnote}{\textbf{Note:~}}
        \begin{quote}
            \rule{\dimexpr\textwidth-2\leftmargini}{1pt}\\
                        \mbox{}\hspace{-\Lnote}\textbf{Note:~}%
                                            #1\\[-0.5ex] 
            \rule{\dimexpr\textwidth-2\leftmargini}{1pt}
        \end{quote}
        \addtolength{\leftmargini}{-4em}}

\newcommand{\ra}[1]{\renewcommand{\arraystretch}{#1}}

\newtheorem{thm}{Theorem}[section]
\newtheorem{prop}[thm]{Proposition}
\newtheorem{lem}[thm]{Lemma}
\newtheorem{cor}[thm]{Corollary}
\newtheorem{defn}[thm]{Definition}
\newtheorem{rem}[thm]{Remark}
\numberwithin{equation}{section}

\newcommand{\homework}[5]{
   \pagestyle{myheadings}
   \thispagestyle{plain}
   \newpage
   \setcounter{page}{1}
   \noindent
   \begin{center}
   \framebox{
      \vbox{\vspace{2mm}
    \hbox to 6.28in { {\bf Ethics for NLP: SS 2024 \hfill {\small (#2)}} }
       \vspace{6mm}
       \hbox to 6.28in { {\Large \hfill #1  \hfill} }
       \vspace{6mm}
       \hbox to 6.28in {  {\it Name: {\rm #3}
        %NetID: {\rm #4} 
        \hfill Matriculation No.: {\rm #5}} }
      \vspace{2mm}}
   }
   \end{center}
   \markboth{#5 -- #1}{#5 -- #1}
   \vspace*{4mm}
}

\newcommand{\problem}[2]{~\\\fbox{\textbf{Problem #1}}\newline\newline}
\newcommand{\subproblem}[1]{~\newline\textbf{(#1)}}
\newcommand{\D}{\mathcal{D}}
\newcommand{\Hy}{\mathcal{H}}
\newcommand{\VS}{\textrm{VS}}

\newcommand{\bbF}{\mathbb{F}}
\newcommand{\bbX}{\mathbb{X}}
\newcommand{\bI}{\mathbf{I}}
\newcommand{\bX}{\mathbf{X}}
\newcommand{\bY}{\mathbf{Y}}
\newcommand{\bepsilon}{\boldsymbol{\epsilon}}
\newcommand{\balpha}{\boldsymbol{\alpha}}
\newcommand{\bbeta}{\boldsymbol{\beta}}
\newcommand{\0}{\mathbf{0}}
\newcommand{\code}[1]{\texttt{#1}}


\begin{document}
\homework{Homework 2}{Due: 19 June 2024, 11:59pm}{Julian Steiner}{}{2669944}

%Read all the instructions carefully before you start working on the assignment, and before you make a submission.

\problem{1}{}
\subproblem{1.1}
\begin{table}[h!]
    \centering
    \caption{3-anonymit Crime Data by State in Germany}
    \label{tab:crime_data}
    \begin{tabular}{ccccccl}
        \toprule
        \textbf{(id)} & \textbf{Name} & \textbf{Age} & \textbf{Gender} & \textbf{State of Germany} & \textbf{Crime} \\ \midrule
        1  & *  & 30-39  & Male   & Rhineland-Palatinate   & Murder   \\ \midrule
        2  & *  & 20-29  & Female & Bavaria                & Murder   \\ \midrule
        3  & *  & 30-39  & Female & North Rhine-Westphalia & Robbery  \\ \midrule
        4  & *  & 30-39  & Male   & Rhineland-Palatinate   & Assault  \\ \midrule
        5  & *  & 20-29  & Female & Bavaria                & Robbery  \\ \midrule
        6  & *  & 30-39  & Female & North Rhine-Westphalia & Murder   \\ \midrule
        7  & *  & 30-39  & Male   & Rhineland-Palatinate   & Parking  \\ \midrule
        8  & *  & 10-19  & Male   & Hesse                  & Murder   \\ \midrule
        9  & *  & 30-39  & Female & North Rhine-Westphalia & Parking  \\ \midrule
        10 & *  & 20-29  & Female & Bavaria                & Speeding \\ \midrule
        11 & *  & 10-19  & Male   & Hesse                  & Robbery  \\ \midrule
        12 & *  & 30-39  & Male   & North Rhine-Westphalia & Assault  \\ \midrule
        13 & *  & 30-39  & Male   & North Rhine-Westphalia & Speeding \\ \midrule
        14 & *  & 10-19  & Male   & Hesse                  & Speeding \\ \midrule
        15 & *  & 30-39  & Male   & North Rhine-Westphalia & Murder   \\ \bottomrule
    \end{tabular}
\end{table}

\subproblem{1.2}

\begin{itemize}
    \item Crime: Murder, Assault, Parking $\Rightarrow$ l-diversity $\Rightarrow$ 3
    \begin{table}[H]
        \centering
        \caption{Group 1 - Crime Data by State in Germany}
        \label{tab:crime_data-1}
        \begin{tabular}{cccccc}
            \toprule
            \textbf{(id)} & \textbf{Name} & \textbf{Age} & \textbf{Gender} & \textbf{State of Germany} & \textbf{Crime} \\ \midrule
            1  & *  & 30-39  & Male   & Rhineland-Palatinate   & Murder   \\ \midrule
            4  & *  & 30-39  & Male   & Rhineland-Palatinate   & Assault  \\ \midrule
            7  & *  & 30-39  & Male   & Rhineland-Palatinate   & Parking  \\ \bottomrule
        \end{tabular}
    \end{table}

    \pagebreak
        
    \item Crime: Murder, Robbery, Speeding $\Rightarrow$ l-diversity $\Rightarrow$ 3
    \begin{table}[H]
        \centering
        \caption{Group 2 - Crime Data by State in Germany}
        \label{tab:crime_data-2}
        \begin{tabular}{cccccc}
            \toprule
            \textbf{(id)} & \textbf{Name} & \textbf{Age} & \textbf{Gender} & \textbf{State of Germany} & \textbf{Crime} \\ \midrule
            2  & *  & 20-29  & Female & Bavaria                & Murder   \\ \midrule
            5  & *  & 20-29  & Female & Bavaria                & Robbery  \\ \midrule
            10 & *  & 20-29  & Female & Bavaria                & Speeding \\ \bottomrule
        \end{tabular}
    \end{table}

    \item Crime: Robbery, Murder, Parking $\Rightarrow$ l-diversity $\Rightarrow$ 3
    \begin{table}[H]
        \centering
        \caption{Group 3 - Crime Data by State in Germany}
        \label{tab:crime_data-3}
        \begin{tabular}{cccccc}
            \toprule
            \textbf{(id)} & \textbf{Name} & \textbf{Age} & \textbf{Gender} & \textbf{State of Germany} & \textbf{Crime} \\ \midrule
            3  & *  & 30-39  & Female & North Rhine-Westphalia & Robbery  \\ \midrule
            6  & *  & 30-39  & Female & North Rhine-Westphalia & Murder   \\ \midrule
            9  & *  & 30-39  & Female & North Rhine-Westphalia & Parking  \\ \bottomrule
        \end{tabular}
    \end{table}

    \item Crime: Murder, Robbery, Speeding $\Rightarrow$ l-diversity $\Rightarrow$ 3
    \begin{table}[H]
        \centering
        \caption{Group 4 - Crime Data by State in Germany}
        \label{tab:crime_data-4}
        \begin{tabular}{cccccc}
            \toprule
            \textbf{(id)} & \textbf{Name} & \textbf{Age} & \textbf{Gender} & \textbf{State of Germany} & \textbf{Crime} \\ \midrule
            8  & *  & 10-19  & Male   & Hesse                  & Murder   \\ \midrule
            11 & *  & 10-19  & Male   & Hesse                  & Robbery  \\ \midrule
            14 & *  & 10-19  & Male   & Hesse                  & Speeding \\ \bottomrule
        \end{tabular}
    \end{table}

    \item Crime: Assault, Speeding, Murder $\Rightarrow$ l-diversity $\Rightarrow$ 3
    \begin{table}[H]
        \centering
        \caption{Group 5 - Crime Data by State in Germany}
        \label{tab:crime_data-5}
        \begin{tabular}{cccccc}
            \toprule
            \textbf{(id)} & \textbf{Name} & \textbf{Age} & \textbf{Gender} & \textbf{State of Germany} & \textbf{Crime} \\ \midrule
            12 & *  & 30-39  & Male   & North Rhine-Westphalia & Assault  \\ \midrule
            13 & *  & 30-39  & Male   & North Rhine-Westphalia & Speeding \\ \midrule
            15 & *  & 30-39  & Male   & North Rhine-Westphalia & Murder   \\ \bottomrule
        \end{tabular}
    \end{table}
\end{itemize}

\notte{From the above equivalence classes, each class has 3 unique crimes. Therefore, the l-diversity of the modified table is 3.}

\pagebreak

\subproblem{1.3}

\begin{table}[H]
    \centering
    \caption{Group 1 - Crime Data by State in Germany}
    \label{tab:crime_data-1-t-closeness}
    \begin{tabular}{cccccc}
        \toprule
        \textbf{(id)} & \textbf{Name} & \textbf{Age} & \textbf{Gender} & \textbf{State of Germany} & \textbf{Crime} \\ \midrule
        1  & *  & 30-39  & Male   & Rhineland-Palatinate   & Murder   \\ \midrule
        4  & *  & 30-39  & Male   & Rhineland-Palatinate   & Assault  \\ \midrule
        7  & *  & 30-39  & Male   & Rhineland-Palatinate   & Parking  \\ \bottomrule
    \end{tabular}
\end{table}

\begin{itemize}
  \item \textbf{Q $\rightarrow$ General Distribution}
  \begin{itemize}
    \item Assault $\rightarrow$ 2/15 $\rightarrow$ 0.133
    \item Murder $\rightarrow$ 5/15 $\rightarrow$ 0.333
    \item Parking $\rightarrow$ 2/15 $\rightarrow$ 0.133
    \item Robbery $\rightarrow$ 3/15 $\rightarrow$ 0.2
    \item Speeding $\rightarrow$ 3/15 $\rightarrow$ 0.2
  \end{itemize}
\end{itemize}

$$
  Q = (0.133, 0.333, 0.133, 0.2, 0.2)
$$

\begin{itemize}
  \item \textbf{P $\rightarrow$ Distribution of equivalence class}
  \begin{itemize}
    \item Assault $\rightarrow$ 1/3 $\rightarrow$ 0.333
    \item Murder $\rightarrow$ 1/3 $\rightarrow$ 0.333
    \item Parking $\rightarrow$ 1/3 $\rightarrow$ 0.333
    \item Robbery $\rightarrow$ 0/3 $\rightarrow$ 0.0
    \item Speeding $\rightarrow$ 0/3 $\rightarrow$ 0.0
  \end{itemize}
\end{itemize}

$$
  P = (0.333, 0.333, 0.333, 0.0, 0.0)
$$

\begin{itemize}
    \item \textbf{t-closeness}
\end{itemize}

$$
D(P,Q) = \sum_{i=1}^m \frac{1}{2} | p_i - q_i | = 
$$
$$
= \frac{1}{2} * (|0.133 - 0.333| + |0.333 - 0.333| + |0.133 - 0.333| + |0.2-0.0| + |0.2 - 0.0|) = 
$$
$$
= \frac{1}{2} * 0.8 = 0.4
$$

\pagebreak

\problem{2}{}

See jupyter notebook for function implementation. The results of the function shown in \ref{tab:differential_privacy_results}.

\begin{table}[ht]
    \centering
    \begin{tabular}{rrrrr}
        \toprule
        \textbf{p} & \textbf{X\_0} & \textbf{X\_1} & \textbf{Y\_0} & \textbf{Y\_1} \\
        \midrule
        0.0 & 0.8 & 0.2 & 0.20 & 0.80 \\ \midrule
        0.2 & 0.8 & 0.2 & 0.32 & 0.68 \\ \midrule
        0.5 & 0.8 & 0.2 & 0.50 & 0.50 \\ \midrule
        0.8 & 0.8 & 0.2 & 0.68 & 0.32 \\ \midrule
        1.0 & 0.8 & 0.2 & 0.80 & 0.20 \\
        \bottomrule
    \end{tabular}
    \caption{}
    \label{tab:differential_privacy_results}
\end{table}

\begin{itemize}
    \item \textbf{Privacy}
    \begin{itemize}
        \item \textbf{p=0 and p=1}:
            \begin{itemize}
                \item When \(p=0\), the noisy distribution \( Y \) is exactly the reverse of the true distribution \( X \). This provides no privacy because the distribution \( Y \) directly reveals the opposite of \( X \).
                \item When \(p=1\), the noisy distribution \( Y \) is identical to the true distribution \( X \). This also provides no privacy because \( Y \) is exactly \( X \).
            \end{itemize}
        \item \textbf{p=0.5}:
            When \(p=0.5\), the noisy distribution \( Y \) becomes uniform (i.e., \( Y=[0.5,0.5] \)). This provides the highest level of privacy because \( Y \) does not reveal any information about the true distribution \( X \).
        \item \textbf{p=0.2 and p=0.8}:
            These intermediate values provide a balance between privacy and utility. As \( p \) moves away from 0.5 towards 0 or 1, the privacy decreases because \( Y \) starts resembling \( X \) more closely.
        \end{itemize}
    \item \textbf{Utility}
    \begin{itemize}
        \item \textbf{p=0 and p=1}:
            These values provide maximum utility because \( Y \) is either exactly \( X \) or exactly the reverse of \( X \). There is a clear correspondence between \( Y \) and \( X \), making \( Y \) highly useful for accurate analysis.
        \item \textbf{p=0.5}:
            When \(p=0.5\), the utility is the lowest because the noisy distribution \( Y \) is uniform and does not reflect the true distribution \( X \) at all. This makes \( Y \) less useful for analysis.
        \item \textbf{p=0.2 and p=0.8}:
            These intermediate values provide a balance between privacy and utility. As \( p \) moves away from 0.5 towards 0 or 1, the utility increases because \( Y \) starts resembling \( X \) more closely.
    \end{itemize}
\end{itemize}

\pagebreak

\problem{3}{}

\begin{itemize}
    \item Text 1

    Had such a blast hanging out with [NAME] and [NAME] in \#[LOC] [DATE]! We explored the city, went to the \#[LOC]Zoo, and had an awesome time. It’s always a good time catching up with old friends and making new memories together. [LOC] has so much character – definitely need to come back soon! Thanks for the awesome day, [NAME] and [NAME]. Let’s do it again sometime! \#Friends \#[LOC] \#FB20 \#TUDA


    \item Text 2
        
        Date: [DATE]
        
        To: [URL], [URL], [URL], [URL]
        
        Title: RE: RE: RE: website
        
        Dear website team,
        
        here is a kick-off mail for the website. Format – a github page which lists some papers, between [URL] (too little info) and [URL] (too much for now). For starters, we’d pick 50 papers from the list we already made [URL]. [NAME] has kindly agreed to lead the work on this. [NAME] has agreed to develop the site. [NAME] and [NAME] have agreed to help with selecting papers and giving feedback. We need to have it running by [DATE] to link it in the paper. With that, I pass it to [NAME].
        
        I guess the first step is setting up the repo and adding all people in this thread as Maintainers. My Github account is [URL]. If there are any questions, please let me know, if something urgent pops up, give me a call at [PHONE].
        
        Best,
        
\end{itemize}

\end{document}
